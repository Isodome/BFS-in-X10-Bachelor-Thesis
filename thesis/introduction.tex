%!TEX root = thesis.tex
%% ==============================
\chapter{Einleitung}
\label{ch:einleitung}
%% ==============================

% Geschwindigkeit durch Parallelität
Lange Zeit wurden Geschwindigkeitssteigerungen der Computer vor allem durch erhöhte Taktraten erreicht. Der Intel 4004 Chip von 1971 taktete mit 108kHz, der 2002 eingeführte Pentium M schon mit 1.7 GHz. 2005 führte Intel den ersten echten Mehrkernprozessor ein(\cite{Intel:2006:Online}), der zwei vollständige Kerne auf einem Chip vereinte. Über 30 Jahre lang optimierte man Prozessoren darauf, einen einzelnen, sequentiellen Befehlsstrang möglichst schnell ausführen zu können. Da in dieser Zeit kein Wechsel des Paradigmas stattfand, skalierten vorhandene Anwendungen sehr gut mit der Geschwindigkeit der Prozessoren. Seit einigen Jahren ist aber klar, dass weitere Geschwindigkeitssteigerungen nur durch (massive) Parallelität geschehen können, da eine weitere Steigerung der Taktraten nur mit sehr hoher Verlustleistung möglich wäre.  

% Kurze BFS Vorstellung
In dieser Arbeit geht es um die Breitensuche. Die Breitensuche (engl: breadth first search, kurz BFS) ist einer der Standardalgorithmen zur Graphtraversierung. Ausgehend von einem Knoten, dem Wurzelknoten, werden alle transitiv erreichbaren Knoten gesucht. Zu jedem der erreichbaren Knoten kann außerdem die Distanz, gemessen in Kantenanzahl und der Vorgängerknoten ausgegeben werden. Die Breitensuche findet Anwendung, wenn der kürzeste Weg von einem Knoten zu allen anderen berechnet werden soll. Eine andere Anwendung ist die Erzeugung des Schichtengraphs, der zum Beispiel in Dinic's Algorithmus zur Lösung des Max-Flow Problems \cite{Dinitz:2006} verwendet wird. 

% Keine unabhängigen Fäden möglich
Die Breitensuche ist ein recht einfacher Algorithmus, dessen sequentielle Version in ein paar Zeilen Code ausgedrückt werden kann. Problematisch ist die Anpassung an parallele Programmierparadigmen, denn es können keine voneinander unabhängigen Aufgaben für einzelne Bearbeitungsfäden definiert werden. Sowohl Daten- als auch Kontrollfluss müssen teuer synchronisiert werden.


% Überblick über die Arbeit
Das Paper von \cite{Buluc:2011} beschreibt Ansätze zur Parallelisierung der Breitensuche. Dabei werden zur Kommunikation vor allem MPI Operationen eingesetzt. Wie die Autoren des Papers bereits bemerken, bleibt die Frage offen, ob die vorgeschlagenen Konzepte auch mit modernen, implizit parallelen Programmiersprachen funktionieren und performant sein können, da eine abstraktere Beschreibungssprache meistens weniger flexibel ist und etwas mehr Overhead benötigt. Dazu wurden verschiedene Datenstrukturen für Graphen getestet und der Algorithmus auf verschiedene Arten an eine PGAS-System angepasst. Weiter wurde erforscht werden, wie gut die Breitensuche mit den Besonderheiten des inasiven Rechnens harmoniert. Dabei wurde vor allem die Asymmetrie der Rechenleistung als Problem betrachtet, was bedeutet, dass die Rechenleistung an einigen Speicherstandorten im Sinne des PGAS-Modell zum Teil bedeutend größer ist, als an anderen.


%TODO: Andreas frage, welche Zeit in der Einleitung
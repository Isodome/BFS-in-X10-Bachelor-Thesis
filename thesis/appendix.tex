%!TEX root = thesis.tex
%%

%% ==============================
%\chapter{Appendix}
%\label{ch:Appendix}
%% ==============================

\appendix
\addchap{Anhang}


\section{Messwerte}
\label{Anhang-Messwerte}
Testplattform 1, alle Messwerte sind in Millisekunden (ms) angegeben. Der Name \nobreak{100kavX} steht für einen Graphen mit 100 000 (100k) Knoten und einem durchschnittlichen Ausgangsgrad von X. Alle Algorithmen verwenden Arraylisten als Datenstruktur. Der invasive Algorithmus wurde mit einem PE pro Place ausgeführt. Zwischen den Iterationen wurde nichts an dem Claim geändert.
%!TEX root = thesis.tex
\begin{center}

\csvautotabular{Laufzeiten/100kav5.csv} \\ \vspace{1cm}
\csvautotabular{Laufzeiten/100kav50.csv} \\ \vspace{1cm}
\csvautotabular{Laufzeiten/100kav100.csv} \\ \vspace{1cm}
\csvautotabular{Laufzeiten/100kav250.csv} \\ \vspace{1cm}
\csvautotabular{Laufzeiten/100kav500.csv} \\ \vspace{1cm}
\csvautotabular{Laufzeiten/100kav700.csv} \\ \vspace{1cm}
\csvautotabular{Laufzeiten/100kav800.csv} \\ \vspace{1cm}
\csvautotabular{Laufzeiten/100kav900.csv} \\ \vspace{1cm}
\csvautotabular{Laufzeiten/100kav1000.csv} \\ \vspace{1cm}
\csvautotabular{Laufzeiten/100kav1200.csv} \\ \vspace{1cm}
\csvautotabular{Laufzeiten/100kav1500.csv} \\ \vspace{1cm}
\csvautotabular{Laufzeiten/100kav2000.csv} \\ \vspace{1cm}
\end{center}
 


% \setcounter{figure}{0}
		
% \begin{figure} [ht]
%   \centering
%    ein Bild
%   \caption{A figure}
%   \label{fig:BPMNBeispiela}
% \end{figure}
%!TEX root = thesis.tex
%%

%% ==============================
%\chapter{Appendix}
%\label{ch:Appendix}
%% ==============================

\appendix
\addchap{Anhang}

\section{Messwerte - Dichte}
\label{Anhang-Messwerte-Dichte}
Alle Messwerte sind in Millisekunden (ms) angegeben. Alle Graphen haben 100 000 Knoten. Der Name \nobreak{avX} steht für einen Graphen mit 100 000 Knoten und einem durchschnittlichen Ausgangsgrad von X. Alle Algorithmen verwenden Arraylisten als Datenstruktur. Der invasive Algorithmus wurde mit einem PE pro Place ausgeführt. Zwischen den Iterationen wurde nichts an dem Claim geändert. Der Knotengrad liegt bei allen Algorithmen zwischen 1 und $\infty$.
%!TEX root = thesis.tex
\begin{center}
\csvautotabular{Laufzeiten/Dichte/100kav5.csv} \\ \vspace{1cm}
\csvautotabular{Laufzeiten/Dichte/100kav50.csv} \\ \vspace{1cm}
\csvautotabular{Laufzeiten/Dichte/100kav100.csv} \\ \vspace{1cm}
\csvautotabular{Laufzeiten/Dichte/100kav250.csv} \\ \vspace{1cm}
\csvautotabular{Laufzeiten/Dichte/100kav500.csv} \\ \vspace{1cm}
\csvautotabular{Laufzeiten/Dichte/100kav700.csv} \\ \vspace{1cm}
\csvautotabular{Laufzeiten/Dichte/100kav800.csv} \\ \vspace{1cm}
\csvautotabular{Laufzeiten/Dichte/100kav900.csv} \\ \vspace{1cm}
\csvautotabular{Laufzeiten/Dichte/100kav1000.csv} \\ \vspace{1cm}
\csvautotabular{Laufzeiten/Dichte/100kav1200.csv} \\ \vspace{1cm}
\csvautotabular{Laufzeiten/Dichte/100kav1500.csv} \\ \vspace{1cm}
\csvautotabular{Laufzeiten/Dichte/100kav2000.csv} \\ \vspace{1cm}
\end{center}
 

\clearpage
\section{Messwerte - Verteilung}
\label{Anhang-Messwerte-Verteilung}
Alle Messwerte sind in Millisekunden (ms) angegeben. Jeder Graph hat 100 000 Knoten und einen durchschnittlichen Knotengrad von 750. Der invasive Algorithmus wurde mit einem PE pro Place ausgeführt. Zwischen den Iterationen wurde nichts an dem Claim geändert.
\begin{center}
\csvautotabular{Laufzeiten/Verteilung/av750_1_10000.sgraph.csv} \\ \vspace{1cm}
\csvautotabular{Laufzeiten/Verteilung/av750_180_3000.sgraph.csv} \\ \vspace{1cm}
\csvautotabular{Laufzeiten/Verteilung/av750_375_1500.sgraph.csv} \\ \vspace{1cm}
\csvautotabular{Laufzeiten/Verteilung/av750_560_1125.sgraph.csv} \\ \vspace{1cm}
\csvautotabular{Laufzeiten/Verteilung/av750_700_801.sgraph.csv} \\ \vspace{1cm}
\end{center}

\clearpage
\section{Messwerte - Größe}
\label{Anhang-Messwerte-Groesse}
Alle Messwerte sind in Millisekunden (ms) angegeben. Jeder Graph hat einen durchschnittlichen Knotengrad von 100, wobei der Knotengrad jedes Knotens zwischen 1 und 500 liegt. Es wurde auf den invasiven Algorithmus verzichtet, da die Hardware größeren Graphen mit diesem Algorithmus nicht gewachsen ist. Auf die Testläuft mit 9 Places wurde ebenso verzichtet.
%!TEX root = thesis.tex
\begin{figure}[h]
\begin{minipage}[t]{8cm} 
\begin{center} 
\tiny\csvautotabular{Laufzeiten/groesse/10k.sgraph.csv}
\end{center} 
\end{minipage} 
\hfill 
\begin{minipage}[t]{7.5cm}
\begin{center} 
\tiny\csvautotabular{Laufzeiten/groesse/100k.sgraph.csv}
\end{center} 
\end{minipage} 
\end{figure}

\begin{figure}[h]
\begin{minipage}[t]{8cm} 
\begin{center} 
\tiny\csvautotabular{Laufzeiten/groesse/200k.sgraph.csv}
\end{center} 
\end{minipage} 
\hfill 
\begin{minipage}[t]{7.5cm}
\begin{center} 
\tiny\csvautotabular{Laufzeiten/groesse/300k.sgraph.csv}
\end{center} 
\end{minipage} 
\end{figure}

\begin{figure}[h] 
\begin{minipage}[t]{8cm} 
\begin{center} 
\tiny\csvautotabular{Laufzeiten/groesse/400k.sgraph.csv}
\end{center} 
\end{minipage} 
\hfill 
\begin{minipage}[t]{7.5cm}
\begin{center} 
\tiny\csvautotabular{Laufzeiten/groesse/500k.sgraph.csv}
\end{center} 
\end{minipage} 
\end{figure}

% \setcounter{figure}{0}
		
% \begin{figure} [ht]
%   \centering
%    ein Bild
%   \caption{A figure}
%   \label{fig:BPMNBeispiela}
% \end{figure}